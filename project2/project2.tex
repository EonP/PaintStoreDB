\documentclass[a4paper,11pt]{article}

% Packages
\usepackage{listings}
\lstset{
    breaklines=true, % Break long lines
    language=SQL, % Set language for syntax highlighting
    basicstyle=\small, % Set the font size
    numbers=none, % No line numbers
    frame=single % Add a frame around the code
}
\usepackage{placeins}
\usepackage{float}
\usepackage{caption}
\usepackage[utf8]{inputenc}
\usepackage{amsmath}
\usepackage{graphicx}
\usepackage{geometry}
\usepackage{enumitem}
\geometry{a4paper, margin=1in}

% Title and Author
\title{Project 2}
\author{Ahmed Tlili, Leon Petrinos, Mathilde Peruzzo}
\date{\today}

\begin{document}

\maketitle

\section{ER Diagram}
\begin{figure}[h]
    \centering
    \includegraphics[width=0.8\textwidth]{ER.png}
    \caption{ER Diagram}
\end{figure}

\section{Relational Schema}
\begin{itemize}
    \item \textbf{Store}(\underline{s\_id}, s\_address, phone\_number, manager\_id UNIQUE NOT NULL)\\
        FOREIGN KEY(manager\_id) REFERENCES Employee(employee\_id)
    \item \textbf{Employee}(\underline{e\_id}, e\_name, s\_id)\\
        FOREIGN KEY(s\_id) REFERENCES Store(s\_id)
    \item \textbf{Manufacturer}(\underline{m\_id}, m\_name)
    \item \textbf{Product}(\underline{p\_id}, p\_name NOT NULL, unit\_price NOT NULL, description,\\ discount\_percentage, m\_id NOT NULL)\\
        FOREIGN KEY(m\_id) REFERENCES Manufacturer(m\_id)
    \item \textbf{Paint}(\underline{p\_id}, base, color)\\
        FOREIGN KEY(p\_id) REFERENCES Product(p\_id)
    \item \textbf{Tool}(\underline{p\_id}, type)
    \item \textbf{Has\_in\_stock}(\underline{p\_id}, \underline{s\_id}, quantity NOT NULL CHECK(quantity $\geq$ 0))\\
        FOREIGN KEY(p\_id) REFERENCES Product(p\_id)\\
        FOREIGN KEY(s\_id) REFERENCES Store(s\_id)
    \item \textbf{Customer}(\underline{email}, c\_name, c\_address NOT NULL)\\
        PRIMARY KEY(email)
    \item \textbf{Purchase}(\underline{p\_id}, amount NOT NULL, p\_date NOT NULL, p\_time NOT NULL)
    \item \textbf{Contains\_purchase}(\underline{p\_id}, \underline{product\_id}, quantity NOT NULL CHECK(quantity $\geq$ 0))\\
        FOREIGN KEY(p\_id) REFERENCES Purchase(p\_id)\\
        FOREIGN KEY(product\_id) REFERENCES Product(p\_id)
    \item \textbf{Instore}(\underline{p\_id}, \underline{e\_id})\\
        FOREIGN KEY(p\_id) REFERENCES Purchase(p\_id)\\
        FOREIGN KEY(e\_id) REFERENCES Employee(e\_id)
    \item \textbf{Online}(\underline{p\_id}, rating CHECK(rating $\geq$ 0 AND rating $\leq$ 5 OR rating IS NULL), delivery\_fee NOT NULL, email NOT NULL)\\
        FOREIGN KEY(p\_id) REFERENCES Purchase(p\_id)\\
        FOREIGN KEY(email) REFERENCES Customer(email)


\end{itemize}

\section{Pending Constraints}
\begin{itemize}
    \item A store should have at least one employee. (TODO: might not be correct as a every store should have a manage which will work there as well)
    \item A purchase should have at least one product.
    \item Cannot have store manager\_id referencing a row in the Employee table.\\
        As here we have two tables referencing each other (STORE, EMPLOYEE).\\
        One of them has to drop the foreign key constraint.
        % -- Add STORE constraint after EMPLOYEE table is created
        % ALTER TABLE STORE
        % ADD CONSTRAINT FK_store_manager FOREIGN KEY(manager_id) REFERENCES EMPLOYEE(e_id);

\end{itemize}

\section{SQL Queries}
\subsection*{Query 1}
\begin{enumerate}[label=(\alph*)]
    \item List the id and address of every store with the respective quantities of the products (with p\_id = 3) they have in stock.
    \item
        \begin{lstlisting}
        SELECT STORE.s_id, s_address, COALESCE(quantity, 0) AS quantity
        FROM STORE
        LEFT JOIN HAS_IN_STOCK
        ON STORE.s_id = HAS_IN_STOCK.s_id AND HAS_IN_STOCK.p_id = 3
        ORDER BY STORE.s_id ASC;
        \end{lstlisting}
    \item
\end{enumerate}
\begin{figure}[H]
    \centering
    \includegraphics[width=0.8\textwidth]{query1.png}
    \caption{Query 1 result}
\end{figure}

\subsection*{Query 2}
\begin{enumerate}[label=(\alph*)]
    \item List the total amount of money spent by each customer in the store with id = 1.\\
        Output should include the customer's email and the total amount of money spent.
    \item
        \begin{lstlisting}
        SELECT CUSTOMER.email, COALESCE(SUM(amount), 0) AS total_amount
        FROM CUSTOMER
        LEFT JOIN ONLINE ON CUSTOMER.email = ONLINE.email
        LEFT JOIN PURCHASE ON ONLINE.p_id = PURCHASE.p_id
        LEFT JOIN STORE ON ONLINE.s_id = STORE.s_id
        GROUP BY CUSTOMER.email
        ORDER BY email ASC;
        \end{lstlisting}
    \item
\end{enumerate}
\begin{figure}[H]
    \centering
    \includegraphics[width=0.8\textwidth]{Query2.png}
    \caption{Query 2 result}
\end{figure}

\subsection*{Query 3}
\begin{enumerate}[label=(\alph*)]
    \item List the id of the biggest in-store purchase made by each store.\\
        Output should include the store id, address and the purchase amount.
    \item
        \begin{lstlisting}
        SELECT STORE.s_id, s_address, COALESCE(MAX(amount), 0) AS max_purchase_amount
        FROM STORE
        LEFT JOIN EMPLOYEE ON STORE.s_id = EMPLOYEE.s_id
        LEFT JOIN INSTORE ON EMPLOYEE.e_id = INSTORE.e_id
        LEFT JOIN PURCHASE ON INSTORE.p_id = PURCHASE.p_id
        GROUP BY STORE.s_id, s_address
        ORDER BY STORE.s_id ASC;
        \end{lstlisting}
    \item
\end{enumerate}
\begin{figure}[H]
    \centering
    \includegraphics[width=0.8\textwidth]{Query3.png}
    \caption{Query 3 result}
\end{figure}

\subsection*{Query 4}
\begin{enumerate}[label=(\alph*)]
    \item List the id and address of every store and the corresponding money ever spent at that store.
    \item
        \begin{lstlisting}
        WITH TEMP_INSTORE AS (
        SELECT STORE.s_id, s_address, COALESCE(SUM(amount), 0) AS total_amount
            FROM STORE
            LEFT JOIN EMPLOYEE ON STORE.s_id = EMPLOYEE.s_id
            LEFT JOIN INSTORE ON EMPLOYEE.e_id = INSTORE.e_id
            LEFT JOIN PURCHASE ON INSTORE.p_id = PURCHASE.p_id
            GROUP BY STORE.s_id, s_address
        ),
        TEMP\_ONLINE AS (
            SELECT STORE.s_id, s_address, COALESCE(SUM(amount), 0) AS total_amount
            FROM STORE
            LEFT JOIN ONLINE ON STORE.s_id = ONLINE.s_id
            LEFT JOIN PURCHASE ON ONLINE.p_id = PURCHASE.p_id
            GROUP BY STORE.s_id, s_address
        )
        SELECT TEMP_INSTORE.s_id, TEMP_INSTORE.s_address,
        TEMP_INSTORE.total_amount + TEMP_ONLINE.total_amount AS total_amount
        FROM TEMP_INSTORE
        LEFT JOIN TEMP_ONLINE ON TEMP_INSTORE.s_id = TEMP_ONLINE.s_id
        ORDER BY TEMP_INSTORE.s_id ASC;
        \end{lstlisting}

    \item
\end{enumerate}
\begin{figure}[H]
    \centering
    \includegraphics[width=0.8\textwidth]{Query4.png}
    \caption{Query 4 result}
\end{figure}

\subsection*{Query 5}
\begin{enumerate}[label=(\alph*)]
    \item List the Paint products that are in that are in maximum quantity in the store with \\id = 1.
        List the product id, name and quantity.
    \item
        \begin{lstlisting}
        WITH TEMP AS (
            SELECT PRODUCT.p_id, p_name, COALESCE(quantity, 0) AS quantity
            FROM PAINT
            LEFT JOIN PRODUCT ON PRODUCT.p_id = PAINT.p_id
            LEFT JOIN HAS_IN_STOCK ON PRODUCT.p_id = HAS_IN_STOCK.p_id
            WHERE HAS_IN_STOCK.s_id = 1
            ORDER BY quantity DESC
        )
        SELECT p_id, p_name, quantity
        FROM TEMP
        WHERE quantity = (SELECT MAX(quantity) FROM TEMP);
        \end{lstlisting}
    \item
\end{enumerate}
\begin{figure}[H]
    \centering
    \includegraphics[width=0.8\textwidth]{Query5.png}
    \caption{Query 5 result}
\end{figure}

\section{SQL Modifications}
\subsection*{Mod 1}
\begin{enumerate}[label=(\alph*)]
    \item Temporarily increase the price of products that where manufactured by the manufacturer with name that ends with "Industries" by 10\%.
    \item
        \begin{lstlisting}
        UPDATE PRODUCT
        SET unit_price = unit_price * 1.1
        WHERE m_id IN (
            SELECT m_id
            FROM MANUFACTURER
            WHERE m_name LIKE '%Industries'
        );
        \end{lstlisting}
    \item
\end{enumerate}
\begin{figure}[H]
    \centering
    \includegraphics[width=0.8\textwidth]{Mod1.png}
    \caption{Mod 1 result}
\end{figure}

\subsection*{Mod 2}
\begin{enumerate}[label=(\alph*)]
    \item Merge two manufacturers with the id1 = 1 and id2 = 2 into a new manufacturer with name "m\_name1-m\_name2".
    \item
        \begin{lstlisting}
        -- Step 1: Insert a new manufacturer with the combined name
        INSERT INTO MANUFACTURER (m_id, m_name)
        SELECT MAX(m_id) + 1,
               (SELECT m_name FROM MANUFACTURER WHERE m_id = 1) || '-' || (SELECT m_name FROM MANUFACTURER WHERE m_id = 2)
        FROM MANUFACTURER;

        -- Step 2: Update products to assign the new manufacturer (with new m_id)
        UPDATE PRODUCT
        SET m_id = (SELECT MAX(m_id) FROM MANUFACTURER)
        WHERE m_id IN (1, 2);

        -- Step 3: Delete the old manufacturers
        DELETE FROM MANUFACTURER WHERE m_id IN (1, 2);
        \end{lstlisting}

    \item
\end{enumerate}
\begin{figure}[H]
    \centering
    \includegraphics[width=0.8\textwidth]{Mod2.png}
    \caption{Mod 2 result}
\end{figure}

\end{document}

